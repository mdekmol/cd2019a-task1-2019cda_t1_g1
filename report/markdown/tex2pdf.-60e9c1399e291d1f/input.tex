\documentclass[12pt,,]{report}
\usepackage{lmodern}
\usepackage{amssymb,amsmath}
\usepackage{ifxetex,ifluatex}
\usepackage{fixltx2e} % provides \textsubscript
\ifnum 0\ifxetex 1\fi\ifluatex 1\fi=0 % if pdftex
  \usepackage[T1]{fontenc}
  \usepackage[utf8]{inputenc}
\else % if luatex or xelatex
  \ifxetex
    \usepackage{mathspec}
  \else
    \usepackage{fontspec}
  \fi
  \defaultfontfeatures{Ligatures=TeX,Scale=MatchLowercase}

    \usepackage{xeCJK}
    % 中文自動換行
    \XeTeXlinebreaklocale "zh"
    % 文字的彈性間距
    \XeTeXlinebreakskip = 0pt plus 1pt
    \newfontlanguage{Chinese}{CHN}
    % 章次20級,節次16級,小節次以下14級,本文12級字
    \def\LARGE{\fontsize{20}{30}\selectfont}%章次
    \def\Large{\fontsize{16}{24}\selectfont}%節次
    \def\large{\fontsize{14}{21}\selectfont}%小節次
    \usepackage{indentfirst}
    \usepackage{CJKnumb}
    \renewcommand{\figurename}{圖}
    \renewcommand{\thefigure}{{\arabic{chapter}}.\arabic{figure}}
    \renewcommand{\tablename}{表}
    \renewcommand{\thetable}{{\arabic{chapter}}.\arabic{table}}
    %重製章節
    \renewcommand{\chaptername}{}
    \renewcommand{\thechapter}{第\CJKnumber{\arabic{chapter}}章}
    \renewcommand{\thesection}{{\arabic{chapter}}.\arabic{section}}
    \renewcommand{\thesubsection}{{\arabic{chapter}}.{\arabic{section}}.\arabic{subsection}}
    %設定行距與中英文字型
    \linespread{1}\selectfont
    \setCJKmainfont{SimSun}
    \setmainfont{Times New Roman}
    \setromanfont{Times New Roman}
    \setmonofont{Times New Roman}
    %重製章節標籤
    \usepackage{titlesec}
    \titleformat{\chapter}[block]{\LARGE\centering}{\thechapter}{0.5em}{}
    \titleformat{\section}[block]{\Large}{\thesection}{0.5em}{}
    \titleformat{\subsection}[block]{\large}{\thesubsection}{0.5em}{}
    % 重製目錄
    \usepackage{titletoc}
    \titlespacing{\chapter}{0pt}{*0}{*2}
    \titlespacing{\section}{0pt}{*1}{*1}
    \titlespacing{\subsection}{0pt}{*1}{*1}
    \titlespacing{\subsubsection}{0pt}{*1}{*1}
    \titlecontents{chapter}[0em]{}{\contentspush{\thecontentslabel}\hspace*{1em}}{}{\titlerule*[0.7pc]{.}\contentspage}
\fi
% use upquote if available, for straight quotes in verbatim environments
\IfFileExists{upquote.sty}{\usepackage{upquote}}{}
% use microtype if available
\IfFileExists{microtype.sty}{
\usepackage{microtype}
\UseMicrotypeSet[protrusion]{basicmath} % disable protrusion for tt fonts
}{}
\usepackage[margin=1in]{geometry}
\usepackage[unicode=true]{hyperref}
\hypersetup{
            pdfauthor={設計二甲 40623101 王馨慧; 設計二甲 40623108 林郁涵; 設計二甲 40623116 楊子毅; 設計二甲 40623117 楊智傑; 設計二甲 40623119 歐宗韋; 設計二甲 40623122 蔡柄澤; 設計二甲 40623129 陳威誠; 設計二甲 40623140 韓希然; 設計二甲 40423157 朱明棈},
            pdfborder={0 0 0},
            breaklinks=true}
\urlstyle{same}  % don't use monospace font for urls
\ifnum 0\ifxetex 1\fi\ifluatex 1\fi=0 % if pdftex
  \usepackage[shorthands=off,main=]{babel}
\else
  \usepackage{polyglossia}
  \setmainlanguage[]{}
\fi
\IfFileExists{parskip.sty}{%
\usepackage{parskip}
}{% else
\setlength{\parindent}{0pt}
\setlength{\parskip}{6pt plus 2pt minus 1pt}
}
\setlength{\emergencystretch}{3em}  % prevent overfull lines
\providecommand{\tightlist}{%
  \setlength{\itemsep}{0pt}\setlength{\parskip}{0pt}}
\setcounter{secnumdepth}{5}
% Redefines (sub)paragraphs to behave more like sections
\ifx\paragraph\undefined\else
\let\oldparagraph\paragraph
\renewcommand{\paragraph}[1]{\oldparagraph{#1}\mbox{}}
\fi
\ifx\subparagraph\undefined\else
\let\oldsubparagraph\subparagraph
\renewcommand{\subparagraph}[1]{\oldsubparagraph{#1}\mbox{}}
\fi

% set default figure placement to htbp
\makeatletter
\def\fps@figure{htbp}
\makeatother


\begin{document}
%Cover Start
\begin{titlepage}
\vspace{1cm}
\begin{center}
\fontsize{36}{54}\selectfont{
    國立虎尾科技大學\par
}
\fontsize{28}{42}\selectfont{機械設計工程系\par}
\fontsize{24}{36}\selectfont{產品協同設計第一組\par}
\vspace{1.5cm}
\fontsize{20}{30}\selectfont{
    手足球\par
    Table Football\par
}
\vspace{\fill}
\fontsize{18}{27}\selectfont{
    學生:\par
    設計二甲 40623101 王馨慧 \par 設計二甲 40623108 林郁涵 \par 設計二甲 40623116 楊子毅 \par 設計二甲 40623117 楊智傑 \par 設計二甲 40623119 歐宗韋 \par 設計二甲 40623122 蔡柄澤 \par 設計二甲 40623129 陳威誠 \par 設計二甲 40623140 韓希然 \par 設計二甲 40423157 朱明棈 \par
    指導教授:嚴家銘\par
}
\vspace{1.5cm}
\fontsize{16}{24}\selectfont{\par}
\end{center}
\vspace{1cm}
\end{titlepage}

\newcommand\frontmatter{
    \cleardoublepage
    \pagenumbering{roman}
}

\newcommand\mainmatter{
    \cleardoublepage
    \pagenumbering{arabic}
}

\newcommand\backmatter{
    \if@openright
        \cleardoublepage
    \else
        \clearpage
    \fi
}

%Document start

% Set page number to arabic i ii...
\frontmatter
\renewcommand{\abstractname}{\LARGE \center 摘要}
\chapter*{摘要}
\addcontentsline{toc}{chapter}{摘要}
\fontsize{14}{21}\selectfont{手足球系統設計

手足球系統模擬

送球機構設計

送球機構模擬

手足球系統功能}


\begingroup
    \renewcommand{\contentsname}{\center 目錄 \addcontentsline{toc}{chapter}{目錄}}
    \renewcommand{\numberline}[1]{~#1\hspace*{1em}}
        \setcounter{tocdepth}{2}
    \tableofcontents
    \newcommand{\lotlabel}{表}
    \renewcommand{\listtablename}{\center 表目錄 \addcontentsline{toc}{chapter}{表目錄}}
    \renewcommand{\numberline}[1]{\lotlabel~#1\hspace*{1em}}
    \listoftables
    \newcommand{\loflabel}{圖}
    \renewcommand{\listfigurename}{\center 圖目錄 \addcontentsline{toc}{chapter}{圖目錄}}
    \renewcommand{\numberline}[1]{\loflabel~#1\hspace*{1em}}
    \listoffigures
\endgroup

% Start normal page number, 1 2 3
\mainmatter
\hypertarget{ux524dux8a00}{%
\chapter{前言}\label{ux524dux8a00}}

\begin{verbatim}
    產品協同的目的就是要一起完成一個商品,在這過程中,每一個人都可以專研自己所擅長的領域,然後透過協同來交換彼此的訊息,分別以不同的研發項目來完成這個作品。

    藉由手足球的模擬,可以實際體會到協同的好處,在有限的時間內,完成超過一個人可以完成的事情,更可以深刻體會到協同之重要性。
\end{verbatim}

\hypertarget{ux8a2dux8a08ux8207ux7e6aux5716}{%
\chapter{設計與繪圖}\label{ux8a2dux8a08ux8207ux7e6aux5716}}

\hypertarget{ux96f6ux7d44ux4ef6ux5c3aux5bf8ux5206ux6790}{%
\section{零組件尺寸分析}\label{ux96f6ux7d44ux4ef6ux5c3aux5bf8ux5206ux6790}}

一、球場分析 球場尺寸依照網頁上的進行設計
.//images/official-foosball-table-dimensions.jpg
而球場高度的部分,以模擬時方便看到球移動為優先考量,取適當高度即可。
.//images/chrome\_8xxV6ukHaq.png

場地變更設計:進行模擬程式時,發現到當球滾到場地角落時,足球員將無法再次擊球,因此參考現實中,足球比賽中所謂的角球的概念將場地邊更成四個角落皆為斜坡,如下圖
.//images/0808080.png

二、球員分析 同樣依照網站上所給隻尺寸進行繪製的動作
.//images/foosball\_player\_dimension.jpg
但由於模擬時,發現球員尺寸會造成兩根桿子平行,球員互相撞擊的部分,因此做了外型上的更改。
.//images/chrome\_mnNsNoqCYo.png

三、桿子分析
桿子我們設定的長度為80in,由於模擬時可能會發生桿子太短,而晃動的情況發生,所以設計長一點來防止這種情況
.//images/chrome\_YlMtfyhoy0.png

四、軌道分析
球進球門後,我們製作一個斜坡讓球能夠停一個角落,等待送球機構把球送到另一個軌道;我們是利用斜坡與重力來運送球。
.//images/球門.png .//images/球門側視圖.png
在這個轉角處的時候,有時候球會卡住,所以在角落處設計一個擋板,這樣能夠確保球不會卡住在這個角落,利用重力能夠繼續滾動。
.//images/縫隙.png .//images/軌道.png
這個孔是將球送回球場,出球孔提高是因為怕球會滾回去而提高的,不會這個孔而影響球的滾動路徑。
.//images/軌道進球場.png

五、成品分析 目前依照前面所設計的圖形,將所有零件組裝完成。
.//images/chrome\_8NEqd8EvlG.png

\hypertarget{ux53c3ux6578ux8a2dux8a08ux8207ux7e6aux5716}{%
\section{參數設計與繪圖}\label{ux53c3ux6578ux8a2dux8a08ux8207ux7e6aux5716}}

一、球檯 .//images/球檯.png

二、球門與軌道 /images/軌道.png /images/球門側視圖.png
.//images/球門.png .//images/縫隙.png

三、球員 .//images/球員.png

四、桿子 .//images/球桿.png

五、送球機構 .//images/送球到高處.png

六、組合 .//images/組合件.png

\hypertarget{ux7d30ux90e8ux8a2dux8a08ux8207-bom}{%
\section{細部設計與 BOM}\label{ux7d30ux90e8ux8a2dux8a08ux8207-bom}}

一、工程圖 1.球場 .//images/球場工程圖.jpg 2.桿子
.//images/桿子工程圖.jpg 3.球員 .//images/球員工程圖.jpg
4.送球機構支撐架 .//images/送球機構支撐架.jpg 5.大風車
.//images/風車工程圖.jpg

\hypertarget{ux9001ux7403ux6a5fux69cbux8a2dux8a08ux8207ux6a21ux64ec}{%
\chapter{送球機構設計與模擬}\label{ux9001ux7403ux6a5fux69cbux8a2dux8a08ux8207ux6a21ux64ec}}

\hypertarget{ux9001ux7403ux6a5fux69cbux8a2dux8a08}{%
\section{送球機構設計}\label{ux9001ux7403ux6a5fux69cbux8a2dux8a08}}

\hypertarget{ux9001ux7403ux6a5fux69cbux6a21ux64ec}{%
\section{送球機構模擬}\label{ux9001ux7403ux6a5fux69cbux6a21ux64ec}}

\hypertarget{ux624bux8db3ux7403ux7cfbux7d71ux6a21ux64ec}{%
\chapter{手足球系統模擬}\label{ux624bux8db3ux7403ux7cfbux7d71ux6a21ux64ec}}

\hypertarget{ux7cfbux7d71ux529fux80fdux5c55ux793a}{%
\chapter{系統功能展示}\label{ux7cfbux7d71ux529fux80fdux5c55ux793a}}

\hypertarget{ux96d9ux4ebaux9375ux76e4ux63a7ux5236ux5c0dux6253}{%
\section{雙人鍵盤控制對打}\label{ux96d9ux4ebaux9375ux76e4ux63a7ux5236ux5c0dux6253}}

https://youtu.be/yKlaM3ONPdU

\hypertarget{ux55aeux4ebaux9375ux76e4ux63a7ux5236ux8207ux96fbux8166ux5c0dux6253}{%
\section{單人鍵盤控制與電腦對打}\label{ux55aeux4ebaux9375ux76e4ux63a7ux5236ux8207ux96fbux8166ux5c0dux6253}}

https://youtu.be/AE9rLeqkIlc

\hypertarget{ux96d9ux96fbux8166ux5c0dux6253}{%
\section{雙電腦對打}\label{ux96d9ux96fbux8166ux5c0dux6253}}

https://youtu.be/L2WIItHtdpo

https://youtu.be/\_fmiNbCI618

第二個網址是加入送球機構的測試

\hypertarget{ux53c3ux8003ux6587ux737b}{%
\chapter{參考文獻}\label{ux53c3ux8003ux6587ux737b}}


\end{document}
